\documentclass[12pt,english,a4paper]{amsart}

\usepackage[utf8]{inputenc}
\usepackage[T1]{fontenc} 
\usepackage{amsmath}
%\usepackage[english]{babel}
\usepackage{amsthm}
\usepackage{amssymb}
\usepackage{amsfonts}
%\usepackage{bbm}
\usepackage{tikz}
\usepackage{geometry}
\usepackage{enumitem}
\usepackage[numbers]{natbib}
\usepackage{xfrac}    
%\usepackage{blindtext}
\usepackage{hyperref}
\usepackage{graphicx}


\newtheorem{theorem}{Theorem}[section]
\newtheorem{proposition}[theorem]{Proposition}
\newtheorem{lemma}[theorem]{Lemma}
\newtheorem{corollary}[theorem]{Corollary}

\theoremstyle{definition}
\newtheorem{definition}[theorem]{Definition}
\newtheorem{example}[theorem]{Example}
\newtheorem{remark}[theorem]{Remark}
\newtheorem{facts}[theorem]{Facts}

\DeclareMathOperator{\neutral}{\mathbf{e}}
\DeclareMathOperator{\N}{\mathbb{N}}
\DeclareMathOperator{\Z}{\mathbb{Z}}
\DeclareMathOperator{\R}{\mathbb{R}}


\newcommand{\Addresses}{{
		\bigskip
		\footnotesize
		
		Jonathan Wassermann, %\textsc{
		%}\par\nopagebreak
		\textit{E-mail address}: \texttt{jonathan.wassermann01@gmail.com}
		\medskip

}}

% Declare your custom math operator here:
\DeclareMathOperator{\Top}{Top}


\title{Ends of Groups}

\author{Jonathan Wassermann}
\date{}
%\includeonly{Preliminaries}
\begin{document}
	\maketitle
	
	\section{Introduction}
	The concept of ends provides a way to capture the “large-scale” topology of spaces and groups. Informally, an end describes a direction in which a space can “go to infinity.” For topological spaces, ends can be understood as the connected components that remain when removing an ascending sequence of compact subsets. In other words, they represent the connected components of the ideal boundary of the space.
	
	In the setting of finitely generated groups, we associate to each group a Cayley graph, which is a locally finite, connected graph encoding the group’s algebraic structure. The ends of a finitely generated group are then defined as the ends of its Cayley graph. Since all Cayley graphs of a  finitely generated group are quasi-isometric, the number of ends is a quasi-isometry invariant, hence, an invariant of the group itself.
	
	In this talk, we will concentrate on determining the number of ends a group can have and will show that a group must have either 0, 1, 2, or infinitely many ends (Theorem \ref{thm:no_group_with_more_than_three_ends}).
	
	\section{Definitions and examples}
	

	\begin{definition}[Ends of topolocical spaces]
		Let $X$ be a topological space. Consider the poset $\text{Comp}(X)$ of compact subsets of $X$ ordered by inclusion. For a compact subspae $K$ we consider the complement $X\setminus K$. Each morphism $K_1 \subset K_2$ induces a function $\Pi_0(X\setminus K_2) \to \Pi_0(X\setminus K_1)$. This yields a presheaf
		$$F: \text{Comp}(X) \to \text{Set}$$
		Define the set of ends $end(X) = \lim F$ to be the limit of this presheaf.
	\end{definition}

	\begin{definition}[Proper rays, 3.3.2 \cite{Guilbault2012}]\label{def:proper_ray_top}
		Let $X$ be a topological space. The we call a continous map $\gamma: [0, \infty) \to X$ proper, if $\gamma^{-1}(C)$ is compact for all compact sets $C\subset X$.
	\end{definition}

	\begin{remark}
		For path-connected CW-complexes, ends may be described as homotopy classes of proper rays, which naturally suggests a purely topological approach to their classification. In this seminar, however, our emphasis will be on geometric group theory. Since the geometric model of a group, its Cayley graph equipped with the standard metric, lies in the full subcategory $\textbf{QMet}$ of quasi-geodesic metric spaces, we will carry out our discussion within $\textbf{QMet}$.
	\end{remark}

	\begin{definition}[Proper rays of geodesic spaces]
		Let $X$ be a geodesic space. A \textit{proper ray} is continous map $\gamma:[0, \infty) \to X$ such that the preimage $\gamma^{-1}(B) \subset [0, \infty)$ of all bounded sets $B\subset X$ is bounded.
	\end{definition}

	\begin{remark}
		This is in fact equivalent to definition \ref{def:proper_ray_top}. One direction is obvious, since compact sets in metric spaces are precisely the closed and boudend subsets. For the other direction one could see, that the closure of a bounded subset is compact.
	\end{remark}

\begin{example}
	The ray $\gamma: [0, \infty)\to \R$ given by $\gamma(t) := \arctan(t)$ is not proper, since the image is bounded ($\gamma([0, \infty)) = [0, \frac{\pi}{2})$).
\end{example}
	
	\begin{definition}[Ends of geodesic spaces]
		Let X be a geodesic space.
		We define an equivalence relation of the set of rays $[0, \infty)\to X$ as follows. Let $\gamma_1, \gamma_2 : [0, \infty) \to X$ be two proper rays, then we define $\gamma_1 \sim \gamma_2$, iff for all bounded $B\subset X$ there exists a $t\geq 0$, such that $\gamma_1([t, \infty))$ and $\gamma_2([t, \infty)$ lie in the same path component of $X \setminus B$.
		We write $\textit{end}(\gamma)$ for such an equivalence class.
	\end{definition}


	\begin{definition}[Space of Ends, Definition 8.2.1 \cite{Loh2017}]
		We define the space of ends as $$\text{End}(X) = \{\textit{end}(\gamma) \mid \gamma :[0, \infty) \to X \text{ proper ray}\}.$$
		We define the topology via convergence:\\
		Let $(x_n)_{n\in \N} \subset \text{End}(X)$ be a sequence of ends and $x\in \text{End}(X)$, then we define $x_n \to x$ to be convergent to $x$, iff there exists representing rays $\gamma_n, \gamma$ of $x_n$ and $x$ respectively with the following property:
		For all bounded sets $B\subset X$ there exists an $N_0 \in \N$ and $t_n \geq 0$ for all $n\geq N_0$ such that $\gamma_n([t_n, \infty)$ and $\gamma([t_n, \infty))$ lie in the same path component of $X\setminus B$.
		Then a subset $A\subset \text{End}(X)$ is closed, if every the limit $x\in \text{End}(X)$ of every converging sequence $(x_n)_{n\in \N} \subset A$ is contained in $A$.
		\end{definition}

	
	\begin{example}
		Consider $X= \R \times \{0\} \cup \{0\}\times \R$. This space has four ends, represented by the rays $t\mapsto (0, t), t\mapsto (0, -t), t\mapsto(0, t), t\mapsto (t, 0)$.
	\end{example}

	\begin{example}
		The space $F_2$ has infinitely many ends. By lemma \ref{lemma_ends_geodesic_spaces}, they are all represented by geodesic rays starting at the origin. Since $F_2$ is the Cayley graph of a free group, i.e. a proper geodesic space, every end is represented by a proper geodesic ray. Such a proper geodesic ray is exactly a sequence of generators without backtracking and hence corresponds to an infinite reduced word.\\
		Fun fact: The set of is uncountable and isomorphic to the Cantor set, see \cite[Sec.~8, p.~236]{Diestel}.
	\end{example}

	\begin{figure}
		\centering
		\includegraphics[width=0.5\linewidth]{Part-of-the-Cayley-graph-of-the-free-group-on-two-group-generators-x-and-y-with-inverses-418933406}
		\caption{Cayley Graph of $F_2$}
		\label{fig:part-of-the-cayley-graph-of-the-free-group-on-two-group-generators-x-and-y-with-inverses-418933406}
	\end{figure}

	
	\begin{definition}[Proper $(c, b)$ quasi ray, Definition 8.2.5 \cite{Loh2017}]
		Let $c\in \R_{>0}, b\in \R_{\geq 0}$ and let $X$ be a $(c,b)$-quasi geodesic space.
		A proper $(c, b)$ \textit{quasi ray} is a proper map $\gamma :[0, \infty)\to X$, such that 
		$$\forall t, t' \geq 0\quad d(\gamma(t), \gamma(t')) \leq c\cdot |t-t'| + b.$$
	\end{definition}

	\begin{definition}[Ends of quasi-geodesic spaces, Definition 8.2.5 \cite{Loh2017}]
			Given two proper $(c, b)$ quasi rays $\gamma_1, \gamma_2: [0, \infty) \to X$. We define the equivalence relation $\gamma_1 \sim \gamma_2$, iff there exists a $t\geq 0$, such that $\gamma_1([t, \infty))$ and $\gamma_2([t, \infty)$ lie in the same $(c, b)$ quasi path component of $X \setminus B$.\\
			We write $\textit{end}_Q(\gamma)$ for the quivalence class of all proper $(c, b)$ quasi rays representing the same quasi end and for the \textit{space of quasi ends} we write 
			$$\text{End}_Q(X) := \{\textit{end}_Q(\gamma) \mid \gamma: [0, \infty) \to X \text{ proper (c, b) quasi ray}\}.$$
			We define the topology on $\text{End}_Q(X)$ via convergence as before by replacing path components by $(c, b)$ quasi path components. 
	\end{definition}


	\begin{proposition}[Quasi-isometry invariance of ends, Prop. 8.2.8, \cite{Loh2017}]
		Let $X, Y$ be quasi geodesic metric spaces, then every quasi isometric embedding $f:X\to Y$ induces a continous map 
		\begin{align*}
			\text{End}_Q(f): \text{End}_Q(X) \to \text{End}_Q(Y) \\
			\text{end}(\gamma) \mapsto 	\text{end}(f\circ \gamma),
		\end{align*}
			i.e. $\text{End}$ induces a functor $\textbf{QMet} \to Top$.
	\end{proposition}

	\begin{lemma}[Ends of geodesic spaces, Prop 8.2.7, \cite{Loh2017}]\label{lemma_ends_geodesic_spaces}
		Let $X$ be a geodesic metric space and $x\in X$. If $X$ is proper, then every end can be represented by a geodesic ray staring in $x$.
	\end{lemma}

	\begin{proof}
		Given such $x\in X$ and an arbitrary ray $\gamma$ representing an end $[\gamma]\in \text{End}(X)$. For every $n\in \N$ connect $x$ and $\gamma(n)$ by a geodesic path $\gamma_n:[0,1]\to X$. Then by Arzela–
		Ascoli theorem we can find a subsequence $\gamma_{n_k}$ that converges to a geodesic ray starting at $x$.
	\end{proof}

	\begin{definition}[Ends of groups, Definition 8.2.9 \cite{Loh2017}]
		Let $G$ be a finitely generated group. Then we define $\text{End}(G) := \text{End}(\Gamma(G, S))$ for some generating set $S\subset G$.
	\end{definition}

	\begin{example}
		Consider the group $\Z^2$. It is easy to see, that the Cayley graph of $\Z^2$ is quasi isometric to $\R^2$. Since ends are invariant under quasi-isometry, it suffices to compute the ends of $\R^2$.
		Now any proper ray $\gamma$ in $\R^2$ represents the same end: 
		
		Let $\gamma'$ be a different proper ray. Let $B\subset X$ by any bounded set. Then $X\setminus B$ is connected. Since $\gamma, \gamma'$ are proper, there exist a $t, t'$, such that $\gamma[t, \infty)\subset X\setminus B$, and $\gamma'[t', \infty)\subset X\setminus B$. Now take $t'' = max(t, t')$.
		
		Geometrically The end space of $\Z^2$ is the one-point compactification of $\R^2$.
	\end{example}
	
	\begin{remark}
		We have seen, that different generating sets of the same group induce quasi isometric Cayley graphs, i.e. isomorphisms in \textbf{QMet}. Hence, by funtoriality this is welldefined.
	\end{remark}

	\begin{proposition}\label{prop_fin_index_subgroup}
		Let $G$ be a finitely geenrated group with finitely many ends. $\text{End}(G)$ inherits a $G$-action by functoriality and $G$ contains a finite index subgroup $H$, such that the restriction to $H$ acts trivial on $\text{End}(G)$.\\
		Moreover $H$ acts trivial on $\text{End}(H)$ and $\text{End}(H) \cong \text{End}(G)$.
	\end{proposition}

	\begin{proof}
		Let $S$ be a finite generating set of $G$. Then $G$ acts on $\Gamma(G, S)$ via	$g.v := g\cdot v$ for a vertex $v\in G$. Now given a ray $\gamma$ in $\Gamma(G, S)$, the group action induces group action via $(g.\gamma)(t) := g\cdot \gamma(t)$. By continuity, this induces a well defined action on $\text{End}(G)$ via $g.[\gamma] := [g\cdot \gamma]$.\\
		The group action induces a morphism of groups
		$$\varphi: G\to \text{Aut}(\text{End}(G)).$$
		Since $G$ is finitely generated, so is the image. By $G/ \ker\varphi \cong \text{im}\varphi$, $H := \ker\varphi$ is of finite index. By definition $\varphi\mid_{\ker\varphi}$ acts trivial on $\text{End}(H)$.\\
		Chose generating sets $T\subset S$, such that $H = \langle T\rangle$ and $G=\langle S\rangle$. The inclusion $H \subset G$ induces a quasi isometry $\Gamma(H, T) \cong \Gamma(G, S)$, since every $g\in G$ lies within a uniformly bounded distance of some $h\in H$ (Can be chosen uniformly and it depends only on the index of $H$).
		Hence, we obtain an isomorphism $\text{End}(H) \cong \text{End}(G)$ and $H$ acts trivial on $\text{End}(H)$ via this isomorphism.
	\end{proof}

	\begin{theorem}[Possible numbers of ends of groups, Theorem 8.2.11, \cite{Loh2017}]\label{thm:no_group_with_more_than_three_ends}

		Let $G$ be a finitely generated group, then G has 0, 1, 2 or $\infty$ many ends.
	\end{theorem}

	\begin{figure}
		\centering
		\includegraphics[width=0.7\linewidth]{no-group-with-three-ends}
		\caption{}
		\label{fig:no-group-with-three-ends}
	\end{figure}

	\begin{proof}
		Assume $G$ has at least three ends.
		By functoriality of $\text{End}$, the space $\text{End}(G)$ inherits a $G$-action and by proposition \ref{prop_fin_index_subgroup} we may without loss of generality assume that the group acts trivial. Let $S$ be a generating set of $G$, $X := \Gamma(G, S)$  and let $\gamma_0, \gamma_1, \gamma_2: [0, \infty) \to X$ be three proper rays representing three different ends. By lemma \ref{lemma_ends_geodesic_spaces} we may assume $\gamma_i$ to be geodesic starting in $e$. Let $r\in \R_{>0}$, such that $\gamma_i(r, \infty)$ lie in three different path components of $X\setminus B_r(e)$. Since $X$ is geodesic, $d(\gamma_1(t), \gamma_2(t')) > 2r$ for all $t, t' > 2r$.\\
		Define $g_k := \gamma_0(k)$ for $k\in \N$. Then $g_k$ acts on theses rays and since the $G$ action is trivial on $\text{End}(G)$, the rays $g_k.\gamma_i$ and $\gamma_i$ represent the same ends. Let $n > 3\cdot r$, then $g_n = \gamma_0(n) \in \gamma_0(r, \infty)$ lies in a different path component of $X\setminus B_r(e)$ than $\gamma_1(r, \infty)$ and $\gamma_2(r, \infty)$. But because $g_n\cdot\gamma_i$ and $\gamma_i$ represent the same end, $g_n\cdot\gamma_i$ has to pass through $B_r(e)$ for $i=1, 2$:
		
		For $t=0$, $g_n\cdot\gamma_i(0) = g_n = \gamma_0(n) \in X\setminus B_r(e)$ lies in the path component of representing the end $[\gamma_0]$. So for $g_n\cdot\gamma_i$ to be in the same component of of $\gamma_i$, it has to pass through $B_r(e)$.
		
		 Now since $n>3\cdot r$ we can find $t_1, t_2 >0$, such that $g_n\cdot\gamma_i(t_i)\in B_r(e)$. 
		Hence:
		\[
		2r < d(\gamma_1(t_1), \gamma_2(t_s)) = d(g_n\cdot \gamma_1(t_1), g_n\cdot\gamma_2(t_2)) \leq r + r = 2r\]
	\end{proof}

	\pagebreak
	
	\begin{remark}
		We can completely classify finitely generated groups with $0$ or $2$ ends.
		
		\smallskip
		
		\noindent\textbf{(1) Zero ends.}
		A finitely generated group $G$ has \emph{no} ends if and only if $G$ is finite.
		
		Indeed, if $G$ is finite, then its Cayley graph is a finite (hence compact) metric space, and no compact space admits a proper ray $[0,\infty)\to X$.  
		Conversely, if $G$ is infinite and finitely generated, then by K\"onig's lemma (\cite[Chapter~8, Theorem~8.2.1]{Diestel}) its Cayley graph $\Gamma(G,S)$ contains an infinite simple path, which is a proper ray; hence $G$ has at least one end.
		
		\smallskip
		
		\noindent\textbf{(2) Two ends.}
		A finitely generated group $G$ has exactly two ends if and only if $G$ is virtually~$\mathbb{Z}$.
		
		\smallskip
		
		\noindent We only sketch the proof. The key ingredients are the following facts:
		\begin{enumerate}
			\item Every connected, locally finite graph with two ends is quasi-isometric to $\mathbb{Z}$.
			\item A finitely generated group that is quasi-isometric to $\mathbb{Z}$ is virtually~$\mathbb{Z}$.
			\item A group that is virtually $\mathbb{Z}$ has two ends.
		\end{enumerate}
		
		Combining these results yields the equivalence:
		\begin{enumerate}
			\item[$(\Rightarrow)$] If $G$ is virtually $\mathbb{Z}$, then it has two ends.
			\item[$(\Leftarrow)$] If $G$ has two ends, then its Cayley graph has two ends, hence is quasi-isometric to $\mathbb{Z}$. Therefore $G$ is virtually $\mathbb{Z}$.
		\end{enumerate}
	\end{remark}
	
	\begin{remark}[Theorem 13.5.5, \cite{Geoghegan2008}]
		There is a very nice classification for ends of groups using group cohomology with coefficents in $\mathbb{F}_2G$, the group ring with coeficients in $\mathbb{F}_2$:
		\[
		\mid\text{End}(G)\mid = 
		\begin{cases}
			0, & \text{if } G \text{ is finite}, \\[6pt]
			1, & \text{if } H^{1}(G;\mathbb{F}_{2}G) = 0, \\[6pt]
			2, & \text{if } \dim_{\mathbb{F}_{2}} H^{1}(G;\mathbb{F}_{2}G) = 1, \\[6pt]
			\infty, & \text{if } \dim_{\mathbb{F}_{2}} H^{1}(G;\mathbb{F}_{2}G) = \infty.
		\end{cases}
		\]
	\end{remark}

	\pagebreak

	\vspace{1cm}
	\begin{thebibliography}{9} % Replace 9 by 99 if you have more that 9 references
		\normalsize
		\bibitem{LastName} Author \emph{Title}, \emph{Journal}, Year
		\bibitem{Loh2017}
		Clara Löh, \emph{Geometric Group Theory: An Introduction}, Universitext, Springer International Publishing, 2017.
		
		\bibitem{Guilbault2012}
		Craig R. Guilbault, \emph{Ends, shapes, and boundaries in manifold topology and geometric group theory}, arXiv:1210.6741 [math.GT], 2012.
		\bibitem{Geoghegan2008}
		R.~Geoghegan,
		\emph{Topological Methods in Group Theory},
		Graduate Texts in Mathematics, vol.~243,
		Springer, New York, 2008.
		\bibitem{Diestel}
		R.~Diestel, \emph{Graph Theory}, 5th ed., Springer, 2017.
		Chapter~8: Infinite Graphs
		Available online at \url{https://diestel-graph-theory.com}.
		
	\end{thebibliography}

	
	
	\Addresses
	
\end{document}

